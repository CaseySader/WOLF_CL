\thispagestyle{plain}
\begin{center}
    \Large
    \textbf{Taming WOLF:}
    
    \vspace{0.4cm}
    \large
    Building a More Functional and User-Friendly Framework
    
    \vspace{0.4cm}
    \textbf{Casey Sader}
    
    \vspace{0.9cm}
    \textbf{Abstract}
\end{center}
Machine learning is all about automation. Many tools have been created to help data scientists automate repeated tasks and train models. These tools require varying levels of user experience to be used effectively. The ``machine learning WOrk fLow management Framework" (WOLF) aims to automate the machine learning pipeline. One of its key uses is to discover which machine learning model and hyper-parameters are the best configuration for a dataset. In this project, features were explored that could be added to make WOLF behave as a full pipeline in order to be helpful for novice and experienced data scientists alike. One feature to make WOLF more accessible is a website version that can be accessed from anywhere and make using WOLF much more intuitive. To keep WOLF aligned with the most recent trends and models, the ability to train a neural network using the TensorFlow framework and Keras library were added. This project also introduced the ability to pickle and save trained models. Designing the option for using the models to make predictions within the WOLF framework on another collection of data is a fundamental side-effect of saving the models. Understanding how the model makes predictions is a beneficial component of machine learning. This project aids in that understanding by calculating and reporting the relative importance of the dataset features for the given model. Incorporating all these additions to WOLF makes it a more functional and user-friendly framework for machine learning tasks.