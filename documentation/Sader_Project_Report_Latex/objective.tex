The goal of this project is to improve the existing WOLF framework so that a user has more tools at their exposure and freedom to do what they feel is needed to learn about their dataset. With WOLF typically having a novice in mind, this project has a focus on what WOLF was lacking that they may want. WOLF does have some limitations and the goal is to improve on some of these that a user would feel are needed in a machine learning framework. The main goal is for WOLF to be user-friendly while still allowing as much customization as the user would like to perform. The three ways this project aims to accomplish this is to create a website version of WOLF, provide access to some of the most used technologies at the moment, and provide the most information to a user as possible. 

The website should have all the capabilities of the command line version of WOLF. The main improvement the website provides is in ease of selecting the configuration for a workflow through a graphical user interface. It also allows a user to create an account and have different WOLF projects associated with their account.

Any good machine learning framework is going to allow the user to train a neural network on their dataset. As TensorFlow is currently one of the most popular libraries for creating deep learning models, this project aimed for learning about the framework and implementing a model in as general a way as possible. 

This project added the capability for a user to have access to the trained model so that they could use it in the future. One of the main ways that one would use a trained model is to make predictions on another similar dataset. A framework to be able to do this was also added. One of the most important tools of machine learning is understanding the importance that each feature of the dataset has in the model. Displaying this information for each model type became a key goal of the project.

In order to show that the neural network and the framework as a whole worked as intended and was feasible, datasets from the UCI Machine Learning repository \parencite{UCIdata} were downloaded and tested. The last goal was to improve the efficiency of WOLF and fix as many bugs as possible.