Within the last few years, machine learning (ML) has become extremely popular and successful in areas such as health, computer vision, finance, speech recognition, sports, and many other fields. Because of the large amount of data in these fields, both experienced researchers in data science and novices that have only just heard about machine learning want to be able to utilize its power. Machine learning algorithms, by definition, automate the rule induction for tasks such as classification and regression that can then be used to better understand a dataset and make predictions on the future or on what actually occurred. There is a lot of work that goes into the data science pipeline besides just fitting a machine learning model and performing all steps, including choosing the best hyper-parameters for the model, can be a cumbersome task. To help with these tasks, WOLF (machine learning WOrk fLow management Framework) was originally created by Pranav Bahl under Dr. Luke Huan in December 2016. This project began on a team under Dr. Huan in 2017 to continue the work and has now become an individual project under Dr. Branicky to keep improving the framework. This project focuses on adding common features that WOLF is lacking and meeting the expected goals of users to make WOLF more user-friendly and accessible. The novice user is the main experience level kept in mind, but an experienced user could also benefit from some of the automation WOLF offers.