In summary, the foundation for a website has been set up and is nearly ready for public use. There has also been a successful implementation of a neural network model that runs on TensorFlow and Keras. Trained models are now saved and able to be used in making predictions within the WOLF framework. The importance of features in relation to the models are also saved and displayed to a user of WOLF to help with interpreting the models. Some extra datasets are available for users to practice on and learn about optimizing WOLF. Also, optimizations have been made to the workflow on the cluster to decrease the time to wait for nodes to run on and increase the efficiency of running on nodes.

This project was a great way to learn about the machine learning pipeline and how to write code that meets industry standards. It is important to create software that allows for new pieces to be easily added and the files in WOLF are purposely created in such a way that a new model or other transaction in the pipeline can be added by following the template that all the existing files follow. This project can be a guide to help future students who would like to further improve WOLF.