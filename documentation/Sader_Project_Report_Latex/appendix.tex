The following are two examples of configuration files for WOLF. The first is the new transaction for making predictions. It shows the required values of the prediction executable script {\tt MakePredictions.py}, the data file to predict with, the desired name of the prediction output ({\tt -o}), and the file path to the pickled model ({\tt -m}). Then there are two optional parameters for the label to predict ({\tt -l}) and whether to compute metrics ({\tt -s}, currently only accuracy) on the predictions. The label name must be given if it is included in the dataset and the metrics can only be calculated if a label name is provided. The second configuration file shows examples of some transactions. These are {\tt datasplit}, {\tt feature\_extraction}, {\tt feature\_selection}, {\tt algorithm}, and {\tt metric\_collection}. Many examples of algorithms and some possible hyper-parameter combinations are also shown.

\begin{figure}[h]
	\centering
	\lstinputlisting[language=sh, basicstyle=\scriptsize]{configurations/wolf_predict.yaml}
	\caption{WOLF configuration file for predictions}
	\label{fig:wolf predict}
\end{figure}

\begin{figure}[h]
	\centering
	\lstinputlisting[language=sh, basicstyle=\scriptsize]{configurations/wolf_presentation.yaml}
	\caption{WOLF configuration file from presentation}
	\label{fig:wolf presentation}
\end{figure}